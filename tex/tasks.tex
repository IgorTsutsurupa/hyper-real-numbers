\begin{task} Получите формулу деления надвещественных чисел, используя их алгебраическую форму.\end{task}
\begin{task} Найдите обратное к $\hat{a}$ число $\hat{a}^{-1}$ ($\hat{a}\hat{a}^{-1}=1$). Для каких $\hat{a}$ оно существует?\end{task}
%\begin{task} Запишите в виде $x+y\hat{o}$: a) $\hat{o}\cdot(a, b)$; b) $\cfrac{1+7\hat{o}}{2+5\hat{o}}$; c) $\cfrac{7\hat{o}}{2+5\hat{o}}$; d) $(0, a-b)$.
%\end{task}
%\begin{task} Определите отношение $<$. Убедитесь, что $<$ и $>$ являются отношениями порядка. \end{task}
%\begin{task} Определим числовое неравенство. Будем говорить, что задано \i{числовое неравенство}, если числа $\hat{a}$ и $\hat{b}$ соеденены отношением порядка $>$ или $<$. Докажите следующие свойства числовых неравенств:
%	\begin{enumerate}[	a)]
%		\item $a > b \Rightarrow \forall c \,\,\, a + c > b + c$;
%		\item $a > b \Rightarrow \forall c > 0 \,\,\, ac > bc$;
%		\item $a > b \Rightarrow \forall c < 0 \,\,\, ac < bc$;
%		\item $a > b, \,\, c > d \Rightarrow a + c > b + d$ (сложение неравенств);
%		\item $a > b, \,\, b \geqslant 0, c > d, d\geqslant 0, \Rightarrow ac > bd$ (умножение неравенств);
%		\item $a > b, \,\, b\geqslant 0, \,\, c < d,\,\,d>0, \,\,c>0 \Rightarrow \cfrac ac > \cfrac bd$ (деление неравенств).
	%\end{enumerate}
%\end{task}