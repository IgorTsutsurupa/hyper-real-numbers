\begin{center}
	\b{ЧАСТЬ 2: ФУНКЦИИ}
\end{center}

Функции $f: \AbR \longmapsto \AbR$ назовем \i{функциями надвещественного переменного}. Поскольку каждое надвещественное число $\hat{\omega}$ характеризуется парой вещественных чисел, то задание надвещественной функции $\hat{\omega}=u+v\hat{o}$ переменной $\hat{z}=x+y\hat{o}$ эквивалентно заданию двух вещественных функций от двух вещественных переменных:
\begin{equation}\label{eq:algebraicfunc}
	f(\hat{z})=u(x, y)+v(x, y)\hat{o},
\end{equation}
Такую запись будем называть алгебраической формой функции $f(\hat{z})$. Функции $u$ и $v$ назовем \i{вещественной} и \i{малой} частями функции $f$ соответственно.
\begin{example}
	Запишем функцию $f(\hat{z})=\hat{z}^3+2\hat{z}^2$ в алгебраической форме. Пусть $\hat{z}=x+y\hat{o}$, тогда:
	\begin{eqnarray*}
		&f(\hat{z})=f(x+y\hat{o})=(x+y\hat{o})^3+2(x+y\hat{o})^2=x^3+3x^2y\hat{o}+2x^2+4xy\hat{o}=      \\ &=(x^3+2x^2)+(3x^2y+4xy)\hat{o}.
	\end{eqnarray*}
\end{example}
\begin{example}
	Найдем область определения этой же функции. Поскольку возведение в степень не определено для чисел $\hat{z}=b\hat{o}$, $b\neq 0$, то 
		$$D(f)=\AbR\textbackslash\{b\hat{o}\,|\,b\neq 0\}.$$
	Попробуем теперь отыскать множество ее значений. Какие значения могут принимать $\Re{f}$ и $\Ab{f}$? В соответствии с \eqref{eq:algebraicfunc} они зависят от $\Re{\hat{z}}=x$ и $\Ab{\hat{z}}=y$. При $x=y=0$ $\Re{f}=\Ab{f}=0$; при $x=0$, $y\neq 0$ $f$ не определена; при $x\neq 0$, $y\in\mathbb{R}$ $\Re{f}$ принимает любые ненулевые значения, $\Ab{f}$ принимает любые значения. Таким образом, 
		$$E(f)=\{0\}\cup\{(a, b)\,|\,a\in\mathbb{R}\textbackslash\{0\},\,b\in\mathbb{R}\}\hm=\AbR\textbackslash\{b\hat{o}\,|\,b\neq 0\}.$$
\end{example}
\b{Многочлены.} \i{Надвещественными многочленами} будем называть суммы вида
	\begin{equation}\label{eq:polynomdef}
		P_n (\hat{z})=\sum_{i=0}^n \hat{c}_i\hat{z}^i,
	\end{equation}
где $\hat{c}_i=a_i+b_i\hat{o}\in\AbR$, $\hat{z}=x+y\hat{o}\in\AbR$. Представим многочлен в алгебраической форме:
	\begin{eqnarray*}
		& P_n(\hat{z})=\sum\limits_{i=0}^n \hat{c}_i\hat{z}^i=\sum\limits_{i=0}^n (a_i+b_i\hat{o})(x+y\hat{o})^i=\sum\limits_{i=0}^n (a_i+b_i\hat{o})(x^i+ix^{i-1}y\hat{o})=\\
		& =\sum\limits_{i=0}^n \big(a_ix^i+(a_iix^{i-1}y+b_ix^i)\hat{o}\big)=\sum\limits_{i=0}^n a_ix^i+\left(\sum\limits_{i=0}^n (a_iix^{i-1}y+b_ix^i)\right)\hat{o}.
	\end{eqnarray*}
	\begin{equation}\label{eq:algebraicpolynom}
		P_n(\hat{z})=\sum_{i=0}^n a_ix^i+\left(\sum_{i=0}^n (a_iix^{i-1}y+b_ix^i)\right)\hat{o}.
	\end{equation}

\b{Дифференциалы.} Рассмотрим функцию $f(\hat{z})$. \i{Правым дифференциалом функции $f(\hat{z})$} назовем функцию надвещественного переменного $g(\hat{z})=f(\hat{z}+\hat{o})-f(\hat{z})$ и обозначим $d_{+}f(\hat{z})$. \i{Левым дифференциалом функции $f(\hat{z})$} назовем функцию надвещественного переменного $g(\hat{z})=f(\hat{z})-f(\hat{z}-\hat{o})$ и обозначим $d_{-}f(\hat{z})$. Далее, если особо не уточняется, под \i{дифференциалом функции} будет подразумеваться ее правый дифференциал и обозначаться просто $df(\hat{z})$.
\begin{example}
	Найдем дифференциал функции $f(\hat{z})=\hat{z}^2$:
		$$df(\hat{z})=f(\hat{z}+\hat{o})-f(\hat{z})=(\hat{z}+\hat{o})^2-\hat{z}^2=\hat{z}^2+2\hat{z}\hat{o}-\hat{z}^2=2\hat{z}\hat{o}.$$
	Если $\hat{z}=x+y\hat{o}$, то
		$$df=2(x+y\hat{o})\hat{o}=2x\hat{o}.$$
\end{example}
\begin{example}
	Найдем дифференциал функции $f(\hat{z})=\hat{z}$:
		$$df(\hat{z})=f(\hat{z}+\hat{o})-f(\hat{z})=\hat{z}+\hat{o}-\hat{z}=\hat{o}.$$
	Это же равенство запишем в виде
		\begin{equation}
			d\hat{z}=\hat{o}.
		\end{equation}
\end{example}
\begin{example}
	Найдем дифференциал функции $f(\hat{z})=\cfrac{1}{\hat{z}}$:
		$$df(\hat{z})=f(\hat{z}+\hat{o})-f(\hat{z})=\cfrac{1}{\hat{z}+\hat{o}}-\cfrac{1}{\hat{z}}=\cfrac{\hat{z}-(\hat{z}+\hat{o})}{(\hat{z}+\hat{o})\hat{z}}=-\cfrac{\hat{o}}{\hat{z}^2+\hat{z}\hat{o}}.$$
	Представим $\hat{z}$ в алгебраической форме:
		$$df=-\cfrac{\hat{o}}{(x+y\hat{o})^2+(x+y\hat{o})\hat{o}}=-\cfrac{\hat{o}}{x^2+(2xy+x)\hat{o}}.$$
	Представим функцию $df$ в алгебраической форме, воспользовавшись формулой деления:
		$$df=-\left(\cfrac{0}{x^2}+\cfrac{x^2\cdot 1-0\cdot(2xy+x)}{x^4}\,\hat{o}\right)=-\cfrac{\hat{o}}{x^2}.$$
	\b{Замечание.} Попробуем найти дифференциал в том виде, в каком мы его понимали раньше:
		$$df=f(x+dx)-f(x)=\cfrac{1}{x+dx}-\cfrac{1}{x}=-\cfrac{dx}{(x+dx)x}.$$
	Если бы мы хотели найти производную функции $f$ делением дифференциала на $dx$, то у нас бы ничего не вышло:
		$$f'=\cfrac{df}{dx}=-\cfrac{1}{x^2+xdx}.$$
	Чтобы получить правильный ответ, приходится совершать предельный переход, отбрасывая слагаемое $xdx$, что мотивируется ``малостью $xdx$ по сравнению с $x^2$''. Надвещественные числа не нуждаются в подобном обхождении, одерживая победу.
\end{example}
\begin{example}
	Обобщим полученные результаты и найдем дифференциал степенной функции $f(\hat{z})=\hat{z}^{\alpha}$, $\alpha\in\mathbb{Q}$:
		\begin{eqnarray*}
			& df=(\hat{z}+\hat{o})^{\alpha}-\hat{z}^{\alpha}=(x+(y+1)\hat{o})^{\alpha}-(x+y\hat{o})^{\alpha}=\\
			& =x^{\alpha}+\alpha x^{\alpha-1}(y+1)\hat{o}-\left(x^{\alpha}+\alpha x^{\alpha-1}y\hat{o}\right),
		\end{eqnarray*}
		\begin{equation}
			d\hat{z}^{\alpha}=\alpha x^{\alpha-1}\hat{o}.
		\end{equation}
\end{example}
\begin{example} 
	Функция $f(\hat{z})=\sgn{\hat{z}}$ для надвещественных чисел определяется так же, как и для вещественных:
		\begin{equation*}
			\sgn{\hat{z}} = \begin{cases}
				1, & \hat{z}>0 \\
				0, & \hat{z}=0 \\
				-1, & \hat{z}<0.
			\end{cases}
		\end{equation*}
	Найдем ее правый дифференциал в точке $\hat{z}=0$. Имеем:
		$$d_{+}\sgn{0}=\sgn{\hat{o}}-\sgn{0}=1-0=1.$$
	Аналогично для левого дифференциала:
		$$d_{-}\sgn{0}=\sgn{0}-\sgn{(-\hat{o})}=0-(-1)=1.$$
	С другой стороны, в любой другой точке $\hat{z}\neq 0$ имеем
		$$d\sgn{\hat{z}}=\sgn{(\hat{z}+\hat{o})}-\sgn{\hat{z}}=0.$$
\end{example}
Последний пример наталкивает на новое определение. Функция $f$ называется \i{дифференцируемой справа в точке $\hat{z}_0$}, если $\Re{d_{+}f(\hat{z}_0)}=0$. Аналогично, функция $f$ называется \i{дифференцируемой слева в точке $\hat{z}_0$}, если $\Re{d_{-}f(\hat{z}_0)}=0$. Функция называется \i{дифференцируемой в точке $\hat{z}_0$}, если она дифференцируема справа и слева, а правый и левый дифференциалы равны. В силу \eqref{eq:algebraicfunc} дифференциал такой функции может быть записан как~$df(x+y\hat{o})\hm=u(x, y)\hat{o}$.
\begin{example}
	Согласно этому определению, функция $\sgn{\hat{z}}$ недифференцируема в точке $\hat{z}=0$, но дифференцируема в любой другой точке. Рассмотрим функцию $f(\hat{z})=|\hat{z}|$, которая задается кусочно:
		\begin{equation*}
			|\hat{z}| = \begin{cases}
				\hat{z}, & \hat{z}\ge 0 \\
				-\hat{z}, & \hat{z}<0 \\
			\end{cases}
		\end{equation*}
	Ее правый дифференциал в точке $\hat{z}=0$ равен
		$$d_{+}|\hat{z}|=|\hat{o}|-|0|=\hat{o},$$
	а левый, с другой стороны,
		$$d_{-}|\hat{z}|=|0|-|-\hat{o}|=-\hat{o}.$$
	Функция дифференцируема слева и справа, но правый и левый дифференциалы не равны, а $|\hat{z}|$ не дифференцируем в нуле.
\end{example}

\b{Свойства дифференциала} Пусть $f$ и $g$ -- функции надвещественного переменного. Тогда
\begin{enumerate}
	\item $df(\hat{z})=0, f(\hat{z})=\hat{z}_0\in\AbR$;
	\item $d(f+g)(\hat{z})=df(\hat{z})+dg(\hat{z})$;
	\item $d(fg)(\hat{z})=f(\hat{z}+\hat{o})dg(\hat{z})+g(\hat{z})df(\hat{z})=f(\hat{z})dg(\hat{z})+g(\hat{z}+\hat{o})df(\hat{z})$;
	\item $d\left({\cfrac{f}{g}}\right)(\hat{z})=\cfrac{g(\hat{z})df(\hat{z})-f(\hat{z})dg(\hat{z})}{g(\hat{z})g(\hat{z}+\hat{o})}$.
\end{enumerate}
Докажем свойства дифференциала по порядку.
\begin{enumerate}
	\item $df(\hat{z})=f(\hat{z}+\hat{o})-f(\hat{z})=\hat{z}_0-\hat{z}_0=0$;
	\item $d(f+g)(\hat{z})=(f+g)(\hat{z}+\hat{o})-(f+g)(\hat{z})=f(\hat{z}+\hat{o})-f(\hat{z})+g(\hat{z}+\hat{o})-g(\hat{z})=df+dg$;
	\item $\begin{aligned}
		& d(fg)(\hat{z})=(fg)(\hat{z}+\hat{o})-(fg)(\hat{z})=\\
		& =f(\hat{z}+\hat{o})g(\hat{z}+\hat{o})-f(\hat{z})g(\hat{z})=\\
		& =f(\hat{z}+\hat{o})g(\hat{z}+\hat{o})-f(\hat{z}+\hat{o})g(\hat{z})+f(\hat{z}+\hat{o})g(\hat{z})-f(\hat{z})g(\hat{z})=\\
		& =f(\hat{z}+\hat{o})\left(g(\hat{z}+\hat{o})-g(\hat{z})\right)+g(\hat{z})\left(f(\hat{z}+\hat{o})-f(\hat{z})\right)=\\
		& =f(\hat{z}+\hat{o})dg(\hat{z})+g(\hat{z})df(\hat{z})
		\end{aligned}$;
	\item $\begin{aligned}
		& d\left({\cfrac{f}{g}}\right)=\cfrac{f(\hat{z}+\hat{o})}{g(\hat{z}+\hat{o})}-\cfrac{f(\hat{z})}{g(\hat{z})}=\\
		& =\cfrac{f(\hat{z}+\hat{o})g(\hat{z})-f(\hat{z})g(\hat{z}+\hat{o})}{g(\hat{z})g(\hat{z}+\hat{o})}=\\
		& =\cfrac{f(\hat{z}+\hat{o})g(\hat{z})-f(\hat{z})g(\hat{z})+f(\hat{z})g(\hat{z})-f(\hat{z})g(\hat{z}+\hat{o})}{g(\hat{z})g(\hat{z}+\hat{o})}= \\
		& =\cfrac{g(\hat{z})\big(f(\hat{z}+\hat{o})-f(\hat{z})\big)-f(\hat{z})\big(g(\hat{z}+\hat{o})-g(\hat{z})\big)}{g(\hat{z})g(\hat{z}+\hat{o})}=\cfrac{g(\hat{z})df(\hat{z})-f(\hat{z})dg(\hat{z})}{g(\hat{z})g(\hat{z}+\hat{o})}.
		\end{aligned}$.
\end{enumerate}

Докажем теорему.
\begin{theorem} 
	Вещественная часть функции $f$ зависит только от вещественной части аргумента тогда и только тогда, когда $f$ дифференцируема:
		\begin{equation}
			\Re{f(\hat{z})}=v(\Re{\hat{z}})\Leftrightarrow df(\hat{z})=u(x, y)\hat{o}.
		\end{equation}
\end{theorem}
\begin{proof}

	\b{Необходимость.} Поскольку $\Re{f(\hat{z})}=v(\Re{\hat{z}})$, то $f(\hat{z})\hm=f(x+y\hat{o})\hm=v(x)+u(x, y)\hat{o}$. Тогда
		\begin{eqnarray*}
			& df(\hat{z})=f(\hat{z}+\hat{o})-f(\hat{z})=f(x+(y+1)\hat{o})-f(x+y\hat{o})=\\
			& =v(x)+u(x, y+1)\hat{o}-\big(v(x)+u(x, y)\hat{o}\big)=\big(u(x, y+1)-u(x, y)\big)\hat{o}=\tilde{u}(x, y)\hat{o}.
		\end{eqnarray*}

	\b{Достаточность.} Пусть функция $f(\hat{z})=v(x, y)+u(x, y)\hat{o}$ дифференцируема. Тогда
		\begin{eqnarray*}
			& df(\hat{z})=f(\hat{z}+\hat{o})-f(\hat{z})=f(x+(y+1)\hat{o})-f(x+y\hat{o})=\\
			& =v(x, y+1)+u(x, y+1)\hat{o}-\big(v(x, y)+u(x, y)\hat{o}\big), \\
			& \big(v(x, y+1)-v(x, y)\big)+\big(u(x, y+1)-u(x, y)\big)\hat{o}=\tilde{u}(x, y)\hat{o}
		\end{eqnarray*}
		Вещественная часть правой части равенства равна нулю, тогда
			$$v(x, y+1)-v(x, y)=0.$$
		Поскольку равенство верно для любых $y$, то $v(x, y)$ зависит только от $x$.
\end{proof}

Мы получили, что дифференцируемые функции записываются в виде $f(x+y\hat{o})\hm=v(x)+u(x, y)\hat{o}$. Отыскать функцию $v(x)$ довольно просто.
\begin{theorem}
	Если $\Re{f(\hat{z})}=v(\Re{\hat{z}})$, то $v(x)=f(x)$.
\end{theorem}
\begin{proof}
	Найдем значение $f$ в вещественной точке $x\in\mathbb{R}$:
		$$f(x)=v(x)+u(x, 0)\hat{o},$$
	из чего немендленно следует, что $f(x)=v(x)$.
\end{proof}

Итак, дифференцируемые функции представимы в виде
	\begin{equation}
		f(\hat{z})=f(x+y\hat{o})=f(x)+u(x, y)\hat{o}.
	\end{equation}